\chapter{Opis projektnog zadatka}
		
		
		Cilj projekta je razviti programsku podršku za stvaranje aplikacije „ParkShare“ koja će korisniku omogućiti rezervaciju i plaćanje parkinga. Korisnik će moći također pregledati sva slobodna parkirališna mjesta za automobile i bicikle. Prilikom pokretanja sustava korisnik će na mapi vidjeti obližnja parkirališta i dobit će rutu do najbližeg te će moći vidjeti ima li slobodnih mjesta kako ne bi došao na popunjeno parkiralište.
Neregistrirani korisnik šalje zahtjev za registraciju s odabranom ulogom koju želi, dakle voditelj parkinga ili klijent. Neregistriranom korisniku pokazuju se dostupna parkirališta i dostupna mjesta, a klijentima se prikazuje i zauzetost mjesta u stvarnom vremenu. Za registraciju korisniku su potrebni sljedeći podaci:

		\begin{packed_item}
			\item \textit{Korisničko ime }
			\item \textit{Lozinka }
			\item \textit{Ime }
			\item \textit{Prezime}
			\item \textit{Slika osobne }
			\item \textit{IBAN račun }
			\item \textit{Email adresa }
		\end{packed_item}
		
	Ukoliko se podaci podudaraju s nekim postojećim korisnikom u bazi podataka sustav šalje grešku korisniku koji se pokušao registrirati i upozorava ga da takav korisnik već postoji. Registracija završava potvrdom putem emaila korisnika, a voditelje mora potvrditi dodatno i administrator. Svaki korisnik aplikacije može vidjeti svoje privatne podatke u aplikaciji i može ih mijenjati. Također korisnik može izbrisati svoj korisnički račun. 
Postoje 3 vrste korisnika:

                \begin{packed_item}
			\item \textit{Klijent }
			\item \textit{Administrator }
			\item \textit{Voditelj parkirališta}
		\end{packed_item}

Administrator može vidjeti popis svih klijenata i voditelja i njihove osobne podatke i može ih mijenjati po volji ako je potrebno. Administrator iz baze podataka dobiva popis svih nepotvrđenih voditelja parkirališta te ih on ukoliko smatra da su im podaci valjani potvrđuje. Administrator može također pregledati sva parkirališta i njihova mjesta u aplikaciji i može mijenjati njihove podatke. Administrator može ukloniti ili promijeniti korisnički račun bilo kojeg ne-administrativnog korisnika. 
Voditelj zapisuje specifikacije svog parkirališta kao što su:

                \begin{packed_item}
			\item \textit{Naziv parkirališta }
			\item \textit{Opis }
			\item \textit{Fotografija}
                        \item \textit{Cjenik}
		\end{packed_item}

Voditelj ucrtava na karti svako parkirališno mjesto svog parkirališta i može za po želji odabrano parkirališno mjesto odrediti može li se rezervirati i postavlja senzore. Ukoliko automobil dođe na određeno parkirališno mjesto gdje se nalazi senzor, senzor javlja da je mjesto sada zauzeto.Ako automobil napusti parkirališno mjesto senzor javlja da se mjesto oslobodilo. Nakon ucrtavanja parkirališnih mjesta zapisuje se ukupan kapacitet parkirališta.
Klijent na karti pregledava parkirališta u svojoj blizini i klikom na njih može vidjeti dostupne informacije o tom parkiralištu ujedno i je li ima slobodnih mjesta na parkiralištu. Ukoliko klijent želi biti siguran, on može rezervirati parkirališno mjesto tako da upiše termin u kojem želi parkirališno mjesto i pokazat će mu sva dostupna parkirališna mjesta na karti koja se mogu rezervirati. Drugi način za rezervaciju je, klijent unaprijed odabere parkirališna mjesta koja želi i onda će mu se prikazati kalendar s dostupnim terminima rezervacije. Klijent može rezervirati mjesto na proizvoljno dugo vremena i može rezervacija biti definirana kao ponavljajuća. Voditelj za svoje parkiralište određuje cijenu ovisno o trajanju rezervacije. Ukoliko se klijent odluči rezervirati parkirališno mjesto, parking mu se odmah naplaćuje putem aplikacije, ako pak klijent dođe bez rezervacije i odabere slobodno parkirališno mjesto, parking se naplaćuje na parkiralištu u tom trenutku. U aplikaciji svaki klijent posjeduje novčanik kojim plaća parking, a u svakom trenutku može uplatiti određenu svotu novca u novčanik kako bi mogao platiti parking. Prilikom naplate parkinga ukoliko klijent nema dovoljno sredstava za platiti na računu, preusmjerava ga se na stranicu za nadoplatu novca u novčanik kako bi mogao podmiriti usluge parkirališta. Službena valuta kojom sustav raspolaže jest Hrvatska kuna - HRK. Svakom korisniku se prikazuje stanje njegovog novčanika i rezervacije koje ima u sustavu parkirališta, administrator nema novčanik i mogućnost rezervacije parkilašnog mjesta pa se njemu prikazuje broj korisnika i broj parkirališta u sustavu.
Voditelji mogu vidjeti statistiku zauzetosti parkirališnih mjesta kroz vrijeme na svom parkiralištu. Prikupljaju se povijesne informacije kako je kroz vrijeme parkiralište bilo zauzeto i statistički se izračunava zauzetost u tom periodu vremena te se voditelju to prikazuje u obliku grafa. Voditelj može ucrtati najviše jedno parkiralište u aplikaciji, a parkirališna mjesta može ucrtavati i brisati po želji, isto tako može izbrisati cijelo parkiralište. Voditelj parkinga može izmijeniti mogućnost rezervacije nekog parkirališnog mjesta te promijeniti cijenu parkinga. 
Napomenuli smo da će na svakom parkiralištu biti moguće ponuditi parking i biciklima. Voditelj parkirališta ne ucrtava parkirališna mjesta kao za automobile već samo određuje ukupan kapacitet parkirališta za bicikle. Korisnici bicikla neće morati brinuti o naplati parkinga jer se on neće naplaćivati takvim korisnicima samim tim neće imati ni opciju za rezervaciju parkinga. Svaki voditelj mora na karti naznačiti je li parkiralište dostupno i za bicikle kako bi korisnici znali na karti gdje mogu parkirati svoje bicikle. Parkirališna mjesta za bicikle neće imati senzor za očitavanje dostupnosti mjesta. U bazu podataka spremamo sve podatke o korisnicima i njihovim ovlastima. Pohranjujemo sve podatke o parkiralištima(ime, cijena, opis), također sve informacije o parkirališnim mjestima(zauzetost, tip). Svaki klijent u aplikaciji nakon korištenja određenog parkirališta dobiva „Like“ opciju, ikonu palčića gore  i dolje te odabire klikom je li mu se parkiralište svidjelo ili ne. Također može i zanemariti danu opciju nije obvezno odgovoriti. Aplikacija mora podržavati rad više korisnika u stvarnom vremenu.


		\eject
		


		
