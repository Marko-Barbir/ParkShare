\chapter{Zaključak i budući rad}
		
		Zadatak nase grupe bio je razvoj web aplikacije za online rezerviranje parkiralisnog mjesta uz mogucnost upravljama korisnicima, voditeljima parkiralista, parkiralistima, parkiralisnim mjestima.
	Nakon 14 tjedana rada u timu i razvoja web aplikacije djelomicno smo uspjeli u ostvarenju naseg cilja. Provedba projekta je tekla kroz dvije faze.
	
	U prvoj fazi smo se okupili kao tim i upoznali, dodijeljen nam je projektni zadatak i radili smo na dokumentaciji projekta.	Zbog dobro provedene prve faze uvelike nam je olaksala daljnji put ka realizaciji web aplikacije.	Izradeni obrasci, dijagrami i baza podataka bili su od velike pomoci svim sudionicima na projektu pri osmisljanju rijesenja i rijesavanju projektnog zadatka.
	
		Druga faza projekta, je bila dosta teza od prve faze zbog nedostatka vremena i slabijeg znanja koristenih alata clanovi tima su gubili dosta vremena pri ucenju samih alata i pri debuggiranju koda. Osim realizacije rjesenja,
		u drugoj fazi je bilo potrebno dokumentirati ostale UML dijagrame i izraditi popratnu dokumentaciju kako bi buduci korisnici mogli lakse koristiti ili vrsiti preinake na sustavu. Clanovi tima su komunicirali putem whatsapp i discord grupa gdje smo obavijestavali ostale clanove o tijeku rada na projektu.
		Sudjelovanje na ovom projektu bilo je vrijedno i na neki nacin naporno iskustvo svim kolegama u timu jer smo kroz intenzivnih nekoliko tjedana rada iskusili zajednicki rad na istom projektu. Zbog losije vremenske organiziranosti imali smo manjih problema no sve u svemu dosta smo naucili i dosta naucenog ce mo koristiti u daljnjoj karijeri.
		Za nastavak rada prvo bi dovrsili cijeli zadatak, a onda bi ugradili i pravi sustav za naplatu.
		
		Potrebno je točno popisati funkcionalnosti koje nisu implementirane u ostvarenoj aplikaciji
		
		\eject 